%----begin preamble----%

\documentclass{article}
\usepackage[utf8]{inputenc}
\usepackage{listings}
\lstset{
   breaklines=true,
   basicstyle=\small\ttfamily,
   columns=flexible}
\pagenumbering{gobble}
\usepackage{geometry}
 \geometry{
 a4paper,
 total={170mm,257mm},
 left=20mm,
 top=20mm,
 }

%----end preamble----%


\begin{document}
\centerline{{\large Notes from Knuth's \TeX book}} \vskip .2cm
\centerline{\it "I can't go to a restaurant and order food because I keep looking at the fonts on the menu. - DK}
\vskip .5cm
1.1 - After you have mastered the material in this book, what will you be: A \TeX pert, or a \TeX nician? \\ \\
I will be the very best \TeX nician, like no one ever was. \\ 

2.1 - Explain how to type the following sentence to \TeX : Alice said, "I always use an en-dash instead of a hyphen when specifying page numbers like '480--491' in a bibliography." 
\begin{lstlisting}
Alice said, "I always use an en-dash instead of a hyphen when specifying page numbers like '480--491' in a bibliography."
\end{lstlisting}

2.2 - What do you think happens when you type four hyphens in a row? \\ \\
My guess is one em-dash and a hyphen. Let's see: ---- \\

2.3 - Think of an English word that contains two ligatures. \\ \\
A {\it ligature} is a combination of letters that are treated as one unit so the word looks better. Two examples would be kerfuffle and filament. \\ 

2.4 - Ok, now you know how to produce \thinspace\rq\rq\thinspace\rq\ and \rq\thinspace\rq\rq; how do you get \lq\lq\thinspace\lq\ and \lq\thinspace\lq\lq\thinspace? 
\begin{lstlisting}
\lq\lq\thinspace\lq\ and \lq\thinspace\lq\lq
\end{lstlisting}

2.5 - Why do you think the author introduced the control sequence \verb|\thinspace| to solve the adjacent-quotes problem, instead of recommending the tricker construction \verb|'$\,$''| (which also works)? \\ \\
Utilizing math mode in this case would be unneccessary. \\

3.1 - What are the control sequences in '\lstinline{\I'm \exercise3.1\\!}'? 
\begin{lstlisting}
\I, \exercise, and \\
\end{lstlisting}

3.2 - We've seen that the input \verb|P\'olya| yields 'P\'olya'. Can you guess how the words 'math\'ematique' and 'centim\`etre' should be specified?
\begin{lstlisting}
math\'ematique and centim\`etre}
\end{lstlisting}

3.3 - Which of the control sequences \verb|\_| and \verb|\(return)| is primitive? 

\verb|\(return)| is primitive because \TeX\ treats this the same as control space.

4.1 - Explain how to type the bibliographic reference 'Ulrich Dieter, {\sl Journal f\"ur die reine und angewandte Mathematik} {\bf 201} (1959), 37--79.' [Use grouping.] \\ 

\lstinline{'Ulrich Dieter, {\sl Journal f\"ur die reine und angewandte Mathematik} {\bf 201} (1959), 37--79.'} \\

4.2 - {\it Explain how to typeset a }roman {\it word in the midst of an italicized sentence.}

\lstinline{Use grouping.} \\ \\

4.4 - Why do you think the author chose the names '\verb|\tenpoint|' and '\verb|\tenrm|', etc., instead of '\verb|\10point|' and '\verb|10rm|'? \\

You can't use numbers in control sequences. 
\newpage
4.5 - Suppose that you have typed a manuscript using slanted type for emphasis, but your editor suddenly tells you to change all the slanted to italic. What's an easy way to do this? 

\begin{lstlisting}
\renewcommand{\sl}{\it}
\end{lstlisting} 

Possibly not the best way to do this, but the simplest. Since \verb|\sl| is already defined, we just redefine it to italics with \verb|\renewcommand|. \\


5.1 - Sometimes you run into a rare word like 'shelfful' that looks better as 'shelfful' without the 'ff' ligature. How can you fool \TeX\ into thinking that there aren't two consecutive f's in such a word?
\begin{lstlisting}
shelf{f}ul
\end{lstlisting}

5.2 - Explain how to get three blank spaces in a row without using '\verb|\_|'.
\begin{lstlisting}
{ { { }}}
\end{lstlisting}

5.3 - What do you think happens if you type the following: 
\begin{lstlisting}
\centerline{This information should be {centered}.}
\centerline So should this.
\end{lstlisting}

The first sentence will be centered, but only the 'S' in the second will be centered.


{\centerline{This information should be {centered}.}}
{\centerline So should this.} \\

5.4 - How about this one?
\begin{lstlisting}
\centerline{This information should be \it centered.}
\end{lstlisting}

Centered with 'centered' in italics.\\
{\centerline{This information should be \it centered.}}


5.5 - Define a control sequence \verb|\ital| so that a user could type 
'\verb|\ital{text}|' 

\begin{lstlisting}
\newcommand{\ital}[1]{{\it #1}\/}
\ital{nice nice nicf}  what \\
{\it nice nice nicf} what
\end{lstlisting}


6.1 Exercise 6.1 - Statistics show that only 7.43 of 10 people who read this manual actually type the \verb|story.tex| file as recommended, but that those people learn \TeX best. So why don't you join them?  \\
\hrule
\vskip 1in
\centerline{\bf A SHORT STORY}
\vskip 6pt
\centerline{\sl by A. U. Thor}
\vskip .5cm
Once upon a time, in a distant galaxy called \"O\"o\c c, there lived a computer named R.~J. Drofnats.

Mr. ~Drofnats---or "R. J.," as he preferred to be called---was happiest when he was at work typesetting beautiful documents. 
\vskip 1in
\hrule
\vskip .5cm
Exercise 6.2 - Look closely at the output of Experiment 2, and compare it to \verb|story.tex|: If you followed the instructions carefully, you will notice a typographical error. What is it, and why did it sneak in? \\
Typed the whole thing on one line so no error :) \\


\end{document}
